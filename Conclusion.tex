\section{Conclusion}

The model presented here is able to accurately simulate the global distribution of ozone, with a slight increase in total ozone column globally compared to observations. The onset of the springtime Antarctic ozone depletion is seen to occur later in the REF-C1 simulation, and the recovery is also delayed compared to observations. 

The slight increase in ozone concentration is also seen in the time series of past ozone depletion. However, the amount of depletion during October from the onset of ozone depletion until 2010 is similar compared to observations and the CCMVal-2 ensemble for both the REF-C2 historical simulation and the historical part of the REF-C2 future projection simulation. There is still a cold bias during October in the stratosphere at 50\,hPa compared to the ERA-Interim data, however, it is not as pronounced as the CCMVal-2 ensemble. The strength of the polar vortex during October agrees well with ERA-Interim data, with slightly weaker westerly winds compared to the CCMVal-2 ensemble. These three diagnostics are very closely related. The fact that a cold bias accompanies a larger amount of ozone indicates that there are some problems within the dynamics of the model. 

Model simulated vertical profiles of ozone compared to Australian ozonesondes for the sites of Melbourne, Macquarie Island and Davis station show very good agreement in the ozone vertical distribution, concentration and variation for the site of Melbourne. There is less agreement for the site of Macquarie Island, most notable during winter, and the altitude of maximum ozone concentration during summer. The site of Davis Station shows the least agreement between the REF-C1 simulation and the ozonesondes, with a significantly higher ozone concentration seen in the model during winter. The vertical location of maximum ozone depletion for this site is also very different. With the makority of ozone depletion occuring 5\,km high in the REF-C1 simulation.

The model shows an increasing SAM index trend for summer and autumn, which indicates a southward shift of mid-latitude winds and stormtracks. During winter and spring there is no noticable trend. The zonal wind differences at 50 hPa show an increasing wind strength between 1979-1998 averages minus 2000-2009 average during December southwards of 60$^\circ$S. This accompanies the increasing summer SAM trend, and agrees well with the ERA-Interim data. There is less agreement with ERA-Interim data during October with a differing dipole in the western hemisphere. This could be due to differing decadal variation or a tighter polar vortex seen in the ERA-Interm data during this period.

The model presented here is able confidently provide an initial contribution from Australia to the chemistry climate modelling community through the CCMI. It is able to reasonably simulate climate diagnostics that have current and are expected to have future impacts on Australian weather and climate, such as modulation of the SAM index. Future versions of this model will follow the Unitek Kingdom UKCA release candidates, with a major goal of obtaining a fully coupled chemistry-climate-ocean model.