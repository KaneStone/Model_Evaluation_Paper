\section{Conclusion}

The model presented here is able to accurately simulate the global distribution of ozone, with a slight increase in total ozone column globally compared to observations. The onset of the springtime Antarctic ozone depletion is seen to occur later in the REF-C1 simulation, and the recovery is also delayed compared to observations. The slight increase in ozone concentration is also seen in the time series of past ozone depletion. Howeverm, the amount of depletion during October from the onset of ozone depletion until 2010 is similar compared to observations and the CCMVal-2 ensemble for both the REF-C2 historical simulation and the historical part of the REF-C2 future projection simulation.

The model presented here is able confidently provide an initial contribution from Australia to the chemistry climate modelling community through the CCMI. It is able to reasonably simulate climate diagnostics that have current and are expected to have future impacts on Australian weather and climate, such as modulation of the SAM index. Future versions of this model will follow the Unitek Kingdom UKCA release candidates, with a major goal of obtaining a fully coupled chemistry-climate-ocean model.