\section{Model Evaluation}
To evaluate the performance of the model in the Southern Hemisphere and the Australian region. We have compared model output to ozonesondes over the Australian sites of Melbourne, Macquarie Island, and Davis station. We have also compared model output data to total column ozone observational data and ERA-Interim temperature and wind data. 

\ref{toc}

Figure 1 shows time series of Antarctic total ozone column, temperature at 50 hPa and zonal winds at 50 hPa for the Reference-C1 and the historical part of the Reference-C2 simulation compared to observations, ERA-Interim data, and the CCMVal-2 ensemble. The Reference-C1 and Reference-C2 simulations are simulating larger values of total ozone column consistently over the entire historical time series compared to observations from the Bodeker Scientific total column ozone database and the CCMVal-2 ensemble. However, the total amount of ozone depletion from 1960 to 2010 is similar compared to observations and the CCMVal-2 ensemble. There is also slight differences between the Reference-C1 and Reference-C2 simulations in the historical period, with the only significant difference between the two simulations before 2005 being the differences in SSTs and SICs as described in the previous section.

Southern Hemisphere summer SAM index is compared against ERA-Interim data. The SAM index was calculated following \cite{DaoyiGong:2007vm}. Using the difference between normalised sea level pressure at 40˚S and 65˚S. Figure 2 shows the SAM index from 1960-2010 for both the reference-C1 and the historical part of the reference-C2 simulations...

\begin{itemize}
\item Plots comparing to Australian ozonesonde sites (ozone and temperature profiles)
\item Ozone total column
\item ozone, temperature, and wind time series for REF-C1 and REF-C2
\item Polar vortex boundary and jet stream
\end{itemize}