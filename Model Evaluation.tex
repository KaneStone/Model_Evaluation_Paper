\section{Model Evaluation}
To evaluate the performance of the model in the Southern Hemisphere and the Australian region. We have compared model output to ozonesondes over the Australian sites of Melbourne, Macquarie Island, and Davis station. We have also compared model output data to total column ozone observational data and ERA-Interim temperature and wind data. 

\subsection{Global ozone}
Figure \ref{fig:toc} shows the zonally averaged total ozone column averaged over 2000-2010 for the REF-C1 hind-cast simulations compared to observations from the total column ozone database. The yearly zonal structure of total ozone column is comparing well with the observations. However, there is consistently more ozone globally within the REF-C1 simulation. The onset of the ozone hole also occurs slightly later compared to the observations, and persists for slightly longer.

\subsection{historical time series}
Figure \ref{fig:time_series} shows time series of Antarctic total ozone column avergaged between 60-90$^{\circ}$, temperature at 50 hPa between 60-90$^{\circ}$, and zonal wind at 50 hPa between 50-70$^{\circ}$ for the REF-C1 and the historical part of the REF-C2 simulation compared to observations, ERA-Interim data, and the CCMVal-2 ensemble. The REF-C1 and REF-C2 simulations are simulating larger values of total ozone column consistently over the entire historical time series compared to observations from the Bodeker Scientific total column ozone database and the CCMVal-2 ensemble. However, the total amount of ozone depletion from 1960 to 2010 is similar compared to observations and the CCMVal-2 ensemble. There is also slight differences between the REF-C1 and REF-C2 simulations in the historical period, with the only significant difference between the two simulations before 2005 being the differences in SSTs and SICs as described in the previous section.

Figure \ref{fig:profile_Melbourne} shows ozone profiles averged between 2000-2010 for the REF-C1 simulation compared to the ozonesonde observations for the site of Melbourne (37.5$^\circ$S, 145$^\circ$E). To highlight the variability, shaded regions show one standard deviation of the monthly averged model output for the REF-C1 profiles and one standard deviation divided by eight for the ozonesonde profiles. For the site of Melbourne, the REF-C1 simulation is agreeing very well with ozonesondes. The peak in ozone concentration is consistent between REF-C1 and ozonesondes throughout all seasons, with the largest difference seen in Spring. The REF-C1 profiles also show a slight consistent increase in ozone concentration compared to ozonesondes over all seasons.

Figure \ref{fig:profile_Macquarie} is the same as fig. \ref{fig:profile_Melbourne} but for Macquarie Island (54.6$^\circ$S, 158.9$^\circ$E). The comparison at this site does not agree as well between the REF-C1 simualtion and ozonesondes. The REF-C1 profiles are simulating a higher maximum ozone concentration altitude during summer and lower during winter. The ozone concentration is also consistently larger for both autumn and winter. The amount of variation is consistent between the two datasets.

Figure \ref{fig:profile_Davis} is the same as fig. \ref{fig:profile_Melbourne} but for Davis station (68.5$^\circ$S, 79$^\circ$E). Comparison of REF-C1 and ozonesonde profiles at this site show the largest differences. During spring the altitude of maximum ozone depletion is around 15\,km for ozonesondes, and around 20\,km for REF-C1. This is also seen during the summer period. The variaibilty during spring for ozonesondes and spring and summer for REF-C1 is also larger than the other seasons. This is due to the variable nature of springtime Antarctic ozone depletion. During autumn and winter REF-C1 is simulating larger ozone concentrations consistent with the other sites. 

Figure \ref{fig:SAM} shows the Southern Hemisphere seasonal SAM indexex for REF-C1 and the historical part of REF-C2.  The SAM index was calculated following \cite{DaoyiGong:2007vm}. Using the difference between normalised sea level pressure at 40˚S and 65˚S. During the summer period, there is a consistent increasing trend. There is also an increasing trend during autumn. During winter and spring. There is no noticable trend.

Figure \ref{fig:anomaly} shows 50\,hPa zonal wind 1979-1988 minus 2000-2009 differences for REF-C1, REF-C2, and ERA-Interim data. For the months of August, October, and December. August shows some small scale differences between the REF-C1 and REF-C2 compared to ERA-Interim. October shows some larger differences, with an opposite dipole in the western hemisphere when comparing REF-C1 and REF-C2 with ERA-Interim. the December differences are very consistent across the REF-C1, REF-C2, and ERA-Interim data. An increasing zonal wind strength is seen south of 60$^\circ$S. This is an indication of the tightening of the polar vortex due to Antarctic ozone depletion.