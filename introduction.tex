\section{Introduction} 

Coupled chemistry climate models are designed to address the interactions between atmospheric chemistry and the Earth-system. This involves the interactions between ozone, and green house gases (GHGs), and the Earth's dynamical climate and weather system. The importance in increasing the understanding these links is very important for the Australian region due to the annual Spring time Antarctic ozone depletion. The Australian region will be effected heavily by these interactions over the course of the next century by the recovery of ozone as well as the changes in GHGs. Thus, simulations, in the form of a global collaboration, such as the currently ongoing Chemistry Climate Model Initiative (CCMI) \citep{Eyring:2013tg} and past chemistry climate modelling projects [references] are very important in understanding the future of Australia's weather and climate.

The annual springtime depletion of Antarctic ozone is attributed to the anthropogenic emissions of ozone depleting substances (ODSs), mostly chlorofluorocarbons (CFCs), the presesce of the Polar vortex, and the formation of polar stratospheric clouds (PSCs) within. In 1987 the Montreal Protocol was signed to fase out the production and release of ODSs into the atmosphere. This has been very effective, with the first significant signs of ozone recovery seen within the last few years [reference WMO]. The Antarctic ozone depletion over the previous half century have been shown to have had a significant impact on Southern Hemisphere climate, mostly through the cooling of the stratosphere by ozone depletion shifting surface wind patterns [reference]. Therefore due to the current onset of ozone recovery, future climate change in the Southern Hemisphere is expected to be influenced by both stratospheric ozone recovery and GHG increases \cite{Arblaster_2011}.

\begin{itemize}
\item Chemistry Climate Modelling and CCMI.
\item Australian chemistry climate modelling in the past
\item Why it is important (link the dynamical and chemical aspects of the climate system, future changes in ozone, etc.)
\item Importance for the Australian region (Shifting of the jet stream. SAM trends – not sure how relevant due to prescribed SSTs?)
\end{itemize}