\section{Introduction} 

Coupled chemistry climate models are designed to address the interactions between atmospheric chemistry and the Earth-system. This involves the interactions between ozone, and green house gases (GHGs), and the Earth's dynamical climate and weather system. Increasing the understanding of these links is very important for the Australian region due to the annual Spring time Antarctic ozone depletion and its role in modulating Southern Hemisphere surface climate. The Australian region will be effected heavily by these interactions over the course of the next century by the recovery of ozone as well as the changes in GHGs. Thus, simulations, in the form of a global collaboration, such as the currently ongoing Chemistry Climate Model Initiative (CCMI) \citep{Eyring:2013tg} and past chemistry climate modelling projects,  are very important in understanding the future of Australia's weather and

The annual springtime depletion of Antarctic ozone is attributed to the anthropogenic emissions of ozone depleting substances (ODSs), mostly chlorofluorocarbons (CFCs), the presence of the polar vortex, and the formation of polar stratospheric clouds (PSCs) within. In 1987, the Montreal Protocol was signed to fase out the production and release of ODSs into the atmosphere. This has been very effective, with the first significant signs of ozone recovery seen within the last few years [reference WMO 2014]. Antarctic ozone depletion over the previous half century has had a significant influence on Southern Hemisphere tropospheric climate, mostly through the cooling of the stratosphere by ozone depletion affecting the Southern Annular Mode (SAM), thus shifting surface wind patterns \citep{Thompson:2011hb}. Other obvious surface impacts are an increase in ultraviolet (UV) radiation reaching the surface \citep{WMO:2011vf}. Therefore due to the current onset of ozone recovery, future climate change in the Australian region is expected to be influenced mostly by stratospheric ozone recovery and changes in GHG concentrations \citep{Arblaster:2011gk}. Future climate change due to anthropogenic emissions of GHGs is also expected to influence stratospheric ozone concentrations. GHG-induced cooling in the upper stratosphere is expected to contribute to an increase in the rate of ozone recovery by slowing gas-phase ozone loss reactions \citep{Barnett:1975eu,Jonsson:2004wb}. A warming troposphere is also predicted to speed up the Brewer-Dobson circulation \citep{Butchart:2006uu}. Thus, the combined effects of a cooler stratosphere and a strengthening of the brewer dobson circulation, causing a speed up of tropical stratospheric ozone advection to mid-latitudes, is expected to cause a smaller recovery trend in tropical stratospheric ozone and a larger recovery trend in the mid-latitudes \citep{Shepherd:2008ws,Li:2009wf}. 

Simulation of these complicated proccesses is needed to properly access the impact of future ozone recovery and changing GHGs on aspects of Australian climate, such as westerly winds and Southern Australian rainfall patterns. The Australian Community Climate and Earth-System Simulator-Chemistry Climate Model (ACCESS-CCM) is used to run historical and future projection, as well as sensitivity simualtions to help address these questions and contribute to the on-going CCMI project. CCMI is designed to bring together the current generation of Coupled Chemistry Climate Models (CCMs) to perform same-type simulations to help address questions relating to chemistry-climate interactions and accompany future ozone assessment reports and Intergovernmental Panel on Climate Change (IPCC) reports. It also follows on from past chemistry climate modelling comparisons, such as, the 2nd Chemistry Climate Modelling Validation (CCMVal-2) \citep{CCMVal:uf}, the Atmospheric Chemistry and Climate Modelling Intercomparison Project (ACCMIP) \citep{Lamarque:2013jm}, and Atmospheric Chemistry and Climate Hindcast (AC&C Hindcast) simulations. 

In this paper we describe the key components of the model we have used in the contribution to CCMI, which marks the first Australian contribution to an international chemistry-climate modelling project. We also describe the two main simulation setups used in this paper for the evaluation of the model. This includes: hind-cast historical simulations and future projection simulations. Evaluation of model performance and analysis of simulation output, focussing on the the Southern Hemisphere, is described in detail with emphasis on model performance for Australian sites and analysis of climate impacts most relevant to the Australian region.