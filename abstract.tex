Chemistry climate models are very important tools for adressing the full interaction of the Earth-system, including the combined roles of greenhouse gases and influencal chemical tracers, such as ozone, on Southern Hemisphere climate and weather. Accurate simulation is becoming increasing important to study the future effects of changing GHGs and stratospheric ozone concentrations on Australian weather. Evaluation of The Australian Community Climate Earth System Simulator-Chemistry Climate Simulator is presented here, focussing on the Southern Hemisphere and the Australian region. This model us used for the Australian contribution to the Chemistry Climate Model Initative, where hind-cast, future projection, and sensitivity simulations are performed as part of a global collaboration. Evaluation of the model shows that it is able to simulate global total ozone column distributions accurately. However, the onset and recovery of springtime Antarctic ozone depletion is delayed slightly when compared to observations for the average period of 2000-2010. Time series of total October averaged column ozone between 60-90$^\circ$S, 50\,hPa temperature between 60-90$^\circ$S, and 50\,hPa zonal wind between 50-70$^\circ$S shows that model simulates a similar amount of ozone depletion during historical period of the simulations compared to observations, but also simulates slightly larger ozone amounts consistently. The 50\,hPa temperature time series shows a cold bias is present compared to ERA-Interim data, however, this is not as pronounced as the past chemistry climate modelling validation projects, such as the 2nd chemistry climate modelling validation. The 50\,hPa zonal wind time series agrees well with the ERA-Interim data.
