Chemistry climate models are very important tools for adressing the full interaction of the Earth-system, including the combined roles of greenhouse gases and influencial chemical tracers, such as ozone, on Southern Hemisphere climate and weather. Accurate simulation is becoming increasing important to study the future effects of changing GHGs and stratospheric ozone concentrations on Australian weather. Evaluation of The Australian Community Climate Earth System Simulator-Chemistry Climate Simulator is presented here, focussing on the Southern Hemisphere and the Australian region. This model is used for the Australian contribution to the Chemistry Climate Model Initative, where historical, future projection, and sensitivity simulations are performed as part of a global collaboration. Evaluation of the model shows that it is able to simulate global total ozone column distributions accurately. However, the onset and recovery of springtime Antarctic ozone depletion is delayed slightly when compared to observations for the average period of 2000-2010. Time series of total October averaged column ozone between 60-90$^\circ$S, 50\,hPa temperature between 60-90$^\circ$S, and 50\,hPa zonal wind between 50-70$^\circ$S show that the model simulates a similar amount of ozone depletion during historical period of the simulations compared to observations, but also simulates slightly larger ozone amounts consistently. The 50\,hPa temperature time series shows a cold bias is present compared to ERA-Interim data, however, this is not as pronounced as the past chemistry climate modelling validation projects, such as the 2nd chemistry climate modelling validation. The 50\,hPa zonal wind time series agrees well with the ERA-Interim data. Comaprison of vertical ozone profiles over the Australian sites of Melbourne, Macquarie Island, and Davis Station with ozonesondes show good agreement for the site of Melbourne, but less agreement for the sites of Macquarie Island and Davis station. With the most noticable difference at Davis station during spring, with the altitude of maximum ozone depletion occuring 5\,km higher, at 20\,km, in the model simulations compared to ozonesondes. Simulations of the southern annular mode (SAM) index shows that there is an increasing SAM trend within the model simulations for the summer and autumn period, and no trend for the winter and spring periods. This is consistent with a poleward shift of the jet stream that accompanies Antarctic ozone depletion and a warming climate. Accompanying the modulations of the SAM, 50\,hPa zonal wind differences of the 1979-1998 average minus 2000-2009 average show a consistent increasing zonal wind strength southward of 60$^\circ$S during December across the model simulations and ERA-Interim data. Evaluation of these model diagniostics shows that the model is able to reasonably simulate the chemistry climate interactions and therefore, along with future model releases, provide an essential tool for assessing and predicting future Australian weather and climate changes.
