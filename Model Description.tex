\section{Model Description}

The model is supplied from the NIWA-UKCA chemistry climate model \citep{Morgenstern:2009bu}. It includes the Unified Model 7.3 (UM 7.3) as its dynamical core with the United Kingdom Chemistry and Aerosols Module (UKCA) as the chemistry component. It also incorporates the Met Office Surface Exchange Scheme-II (MOSES-II). The model setup does not currently incorporate an interactive coupled ocean model, therefore, prescibed sea surface temperatures (SST) and sea ice concentrations (SIC) are used. The model is run in N48 (3.75$^\circ$ by 2.5$^\circ$) horisontal resolution and L60 (60 hybrid height levels) vertical resolution with a model top at 84 km.

The UM 7.3 has a non-hydrostatic setup \citep{Davies:2005vu} and a semi-lagrangian advection scheme \citep{Priestley:1993ur}. gravity wave drag is made up of both an orographic gravity wave drag component \citep{Webster:2003vf} and a parameterised spectral gravity wave drag component \citep{Scaife:2002vt}. Radiation is described by \cite{Edwards:1996wo} and has nine bands in the long-wave part of the spectrum ranging from 3.3e^{-6}\,m to 1.0e^{-2}\,m and six bands in the short-wave part of the spectrum ranging from 2e^{-7}\,m to 1e^{-5}\,m

The UKCA module includes both stratospheric and tropospheric chemistry with 90 chemical species, This includes species involved in $O_x$, $NO_x$, $HO_x$, $BrO_x$, and $ClO_x$ chemistry. Appropriate species undergo dry and wet deposition. The chemical species undergo over 300 reactions, inlcuding bimolecular, termolecular, photolysis, and heterogeneous reactions on polar stratospheric clouds. Photolysis reactions are calculated by the FASTJX scheme \citep{Neu:2007wi}

The model runs evaluated in this paper are also a part of CCMI. This includes a CCMI hind-cast run, labelled REF-C1 from 1960-2010. The historical part of a future projection run, labelled REF-C2 from 1960-2010. For the REF-C1 run, SST and SIC are from the Hadley Centre HaDISST dataset \citep{Rayner:2003ty}. GHGs are from \citep{Meinshausen:2011is} \citep{Riahi:2011dk} and follow RCP 8.5 after 2005. ODSs follow the A1B scenario from \cite{WMO:2011vf}. Anthropogenic and biofuel emmisions follow \citep{Granier:2011dw}. Biomass burning emissions follow \citep{Lamarque:2011wr} \citep{vanderWerf:2006gi} \citep{Schultz:2008wf}. For the REF-C2 run, the only change before 2005 is SST and SIC are taken from a HadGEM2-ES r1p1i1 CMIP5 model run \cite{Jones:2011ii}. After 2005, all forcings follow RCP 6.0. 