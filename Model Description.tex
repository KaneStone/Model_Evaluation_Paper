\section{Model Description}

The model is supplied from the NIWA-UKCA chemistry climate model \citep{Morgenstern:2009bu}. It includes the Unified Model 7.3 (UM 7.3) as its dynamical core with the United Kingdom Chemistry and Aerosols Module (UKCA) as the chemistry component. It also incorporates the Met Office Surface Exchange Scheme-II (MOSES-II). The model setup does not currently incorporate an interactive coupled ocean model, therefore, prescibed sea surface temperatures (SST) and sea ice concentrations (SIC) are used. The model is run in N48 (3.75$^\circ$ by 2.5$^\circ$) horisontal resolution and L60 (60 hybrid height levels) vertical resolution with a model top at ~84 km.


The UM 7.3 has a non-hydrostatic setup \citep{Davies:2005vu} and a semi-lagrangian advection scheme \citep{Priestley:1993ur}. gravity wave drag is made up of both an orographic gravity wave drag component \citep{Webster:2003vf} and a parameterised spectral gravity wave drag component \citep{Scaife:2002vt}. Radiation is described by \cite{Edwards:1996wo} and has nine bands in the long-wave part of the spectrum () and six bands in the short-wave part of the spectrum ().
List model components including domain

UCKA
\begin{itemize}
\item species
\item reactions
\item FASTJX photolysis scheme
\end{itemize}
Setup for CCMI
\begin{itemize}
\item SST are HadISST for hind-cast (Rayner et al., 2003)
\item SST are CMIP5 HADGEM-ES2 for future projection. (Jones et al., 2011)
\item GHGs (Meinshausen, 2011, Raihi et al., 2011)
\item ODSs (WMO, 2011)
\item Anthopogenic and biofuel emissions (Granier et al., 2011)
\item Biomass burning (Lamarque et al., 2010; van der Werf et al., 2006; Schultz et al., 2008)
\item performed
\end{itemize}


